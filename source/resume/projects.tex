%-------------------------------------------------------------------------------
%	SECTION TITLE
%-------------------------------------------------------------------------------
\cvsection{Projects}
%-------------------------------------------------------------------------------
%	CONTENT
%-------------------------------------------------------------------------------
\begin{cventries}

%---------------------------------------------------------
  \cventry
    % https://github.com/Zhayu517/Optimizing-minimal-counterexamples/blob/main/Seminar.pdf
    {Sole Developer \href{https://github.com/Zhayu517/Software-Security-Seminar}{[Code]} \href{https://github.com/Zhayu517/Optimizing-minimal-counterexamples/blob/main/Seminar.pdf}{[Presentation]}} % Role in Project
    {Optimizing Minimal Counterexamples} % Project Name
    {Manchester, UK} % Location
    {Mar. 2022 - May. 2022} % Date(s)
    {
      \begin{cvitems} % Description(s) of tasks/responsibilities
        \item {Help users better understand bugs (Identify and locate errors) in java programs by reducing the numerical size of counterexamples generated by java BMC (Bounded Model Checker)}
        \item {The feasibility of reducing the counterexample value was verified by using two java BMCs, JPF (Java PathFinder) and JBMC (Java Bounded Model Checker)}
        \item {Configure and test various error types such as deadlock, race conditions, overflow, underflow in the gradle environment}
        \item {All benchmarks in SV-COMP were checked by employing JBMC}
      \end{cvitems}
    }

%---------------------------------------------------------
  \cventry
    % https://github.com/Zhayu517/Question-Classifier
    {Implementer \href{https://github.com/Zhayu517/Question-Classifier}{[Code]} \href{https://github.com/Zhayu517/Resume/blob/main/source/resume/Question\%20Classifier.pdf}{[Paper]}} % Role in Project
    {Question Classifier} % Project Name
    {Manchester, UK} % Location
    {Feb. 2022 - Mar. 2022} % Date(s)
    {
      \begin{cvitems} % Description(s) of tasks/responsibilities
        \item {A question classifier written in Python, which accepts a question and output one of N            predefined classes}
        \item {Used Bag of Words (Bow) model, Bidirectional LSTM and its ensemble networks to implement}
        \item {Test under random and pre-trained embeddings}
        \item {Reached at accuracy 0.852 and F1-score 0.697}
      \end{cvitems}
    }

%---------------------------------------------------------
  \cventry
    % https://github.com/Redcxx/leaderfollower
    % presentation 在电脑或者硬盘的AI and games的sem2 文件夹里
    {Implementer \href{https://github.com/Zhayu517/Resume/blob/main/source/resume/Leader\%20Follower.pdf}{[Presentation]}} % Role in Project
    {Leader Follower} % Project Name
    {Manchester, UK} % Location
    {Mar. 2021 - May. 2021} % Date(s)
    {
      \begin{cvitems} % Description(s) of tasks/responsibilities
        \item {A program which plays repeated 2-person Stackelberg pricing games as the leader
               under imperfect information}
        \item {The leader chooses his strategy to play based on the provided set of historical data (a          list of prices given by leaders and followers in the past) and update his knowledge by           analysing the follower's pricing response}
        \item {Used all historical data approach, Modified Moving Window approach and Weighed Least             Square with Forgetting Factor approach}
        \item {Largest profit reached 19.488 (3 d.p.) per trade against 3 different followers}
      \end{cvitems}
    }

%---------------------------------------------------------
  \cventry
    {Sole Developer \href{https://github.com/Zhayu517/Resume/blob/main/source/resume/Road\%20Tracking\%20in\%20Aerial\%20Images.pdf}{[Paper]}} % Role in Project
    {Road Tracking in Aerial Images} % Project Name
    {Manchester, UK} % Location
    {Sep. 2020 - Apr. 2021} % Date(s)
    {
      \begin{cvitems} % Description(s) of tasks/responsibilities
        \item {Without the use of GPS, by analyzing the satellite images returned by the drone, extracting location information surround it (match similar objects on the map to locate the camera) to establish the scope of the drone's activities}
        \item {Using Jupyter Notebook to implement OpenCV, Keras model and U-Net machine learning algorithm under Python to locate the drone}
        \item {Accuracy is around 100 meters for well-trained models}
      \end{cvitems}
    }

%---------------------------------------------------------
  \cventry
    {Implementer \href{https://github.com/Redcxx/KalahPlayer}{[Code]}} % Role in Project
    {AI plays Mancala} % Project Name
    {Manchester, UK} % Location
    {Oct. 2020 - Dec. 2020} % Date(s)
    {
      \begin{cvitems} % Description(s) of tasks/responsibilities
        \item {Mancala (also known as Kalah) is a 2-person board game which is already proved to be a solved game with a first-player win if both players play perfect games}
        \item {Using Monte Carlo tree search and alpha-beta pruning to create an AI to play the perfect game under the pie rule}
        \item {The AI player successfully beat other 37 agents in a tournament of 51 teams}
      \end{cvitems}
    }

%---------------------------------------------------------
  \cventry
    % https://gitlab.cs.man.ac.uk/comp23412_2019/eventlite_F15
    {Implementer} % Role in Project
    {EventLite Website} % Project Name
    {Manchester, UK} % Location
    {Feb. 2020 - May. 2020} % Date(s)
    {
      \begin{cvitems} % Description(s) of tasks/responsibilities
        \item {Eventlite is a website for Event Management Application, designed and built by a team of 6}
        \item {Developed in Java, REST APIs and the Spring Framework}
        \item {Allow users to manage and edit upcoming events in terms of adding/removing attendees, changing location/time of the event, sending automatic emails to attendees when anything updates}
        \item {Connect with Google Map API to visulise the location of the event}
        \item {Code versioned and preserved via GitLab}
      \end{cvitems}
    }

%---------------------------------------------------------
  \cventry
    % https://gitlab.cs.man.ac.uk/comp23311_2019/stendhal_S1Team35
    {Implementer} % Role in Project
    {Stendhal Local Enhancement} % Project Name
    {Manchester, UK} % Location
    {Nov. 2019 - May. 2020} % Date(s)
    {
      \begin{cvitems} % Description(s) of tasks/responsibilities
        \item {Added new cheating commands in the Stendhal game: teleportation, resurrection, invincibility, auto-following, auto-pathfinding}
        \item {Expanded maps on the basis of the original game: added two new accessible scenes}
        \item {Modified attributes of the original items: life recovery of food, attack power of weapons, defense power of clothing, etc.}
        \item {REST API/TDD and Web/SPRING frameworks in Java are used}
      \end{cvitems}
    }

%---------------------------------------------------------
  \cventry
    % https://gitlab.cs.man.ac.uk/k05792mv/BitTime
    {Implementer} % Role in Project
    {BitTime Website} % Project Name
    {Manchester, UK} % Location
    {Jan. 2019 - Apr. 2019} % Date(s)
    {
      \begin{cvitems} % Description(s) of tasks/responsibilities
        \item {BitTime is a website for reminding people that their deadlines are approaching}
        \item {Simple sign up and log in functionality, allows users to add and set their works' titles         and deadlines}
        \item {A google chrome add-on to push notifications if deadlines are about to reach}
        \item {Pure html, css, php, mysql and javascript are used}
      \end{cvitems}
    }

%---------------------------------------------------------

\end{cventries}
